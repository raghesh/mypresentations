\documentclass[9pt]{beamer}
\usepackage{caption}
\usepackage{subcaption}
\usepackage{semantic}
%\usepackage{stmaryrd}
\usepackage{amsthm}
\usepackage{graphicx}
\usepackage{url}

\usepackage{tikz}
\usepackage{amsmath}
\usepackage{courier}
\usepackage{enumerate}
\usepackage{ragged2e}
\usepackage{fixltx2e}
\setbeamertemplate{itemize items}[circle]
\setbeamertemplate{navigation symbols}{}
%\setbeamertemplate{footline}[frame number]
\setbeamertemplate{footline}{
            \raisebox{5pt}{\makebox[\paperwidth]{\hfill\makebox[10pt]{\scriptsize\insertframenumber}}}}
\usepackage{listings}

\lstdefinestyle{footnote}{mathescape,captionpos=b, basicstyle=\footnotesize\ttfamily,
numbers=left, numberstyle=\tiny, xleftmargin=5em}
\lstdefinestyle{normal}{mathescape,captionpos=b, basicstyle=\normalsize\ttfamily,
numbers=left, numberstyle=\tiny, xleftmargin=5em}
\lstdefinestyle{tiny}{mathescape,captionpos=b, basicstyle=\tiny\ttfamily,
numbers=left, numberstyle=\tiny, xleftmargin=5em}
\lstdefinestyle{footnote1}{mathescape, basicstyle=\footnotesize\ttfamily}
\lstdefinestyle{tiny1}{mathescape, basicstyle=\tiny\ttfamily}
\lstdefinestyle{script1}{mathescape, basicstyle=\scriptsize\ttfamily}


%\captionsetup[subfigure]{labelformat=empty} 
%\DeclareCaptionLabelFormat{no-parens}{\textbf{(#2)}}
%\captionsetup[sub]{              % The subcaption settings
%    font=footnotesize,           % Make the font smaller for both label and text
%    textfont=sl,                 % Make only the caption text slanted
%    labelformat=no-parens}        % Use our custom label format

%\renewcommand\thesubfigure{(\alph{subfigure})}

\def\openmp{\mbox{OpenMP}}
\def\pfloop{\mbox{parallel-for-loop}}
\def\pi{\mbox{agent}}
\def\pis{\pi\mbox{s}}
\def\ompsection{\mbox{parallel-section}}
\def\ompsections{\ompsection\mbox{s}}
\def\pfloops{\pfloop\mbox{s}}
\def\lopenmp{\mbox{LOpenMP}}
\def\canIEnterA{canEnter}
\def\canIEnter{\mbox{\tt \canIEnterA{}}}
\def\iexitA{iExit}
\def\iexit{\mbox{\tt \iexitA{}}}
\def\gcount{\mbox{\tt Rdy}}
\def\grpCnt{\mbox{\tt grCnt}}
\def\pfCnt{\mbox{\tt pfCnt}}
\def\isDoneA{{Done}}
\def\isDone{\mbox{\tt \isDoneA{}}}
\def\lpfCnt{\mbox{\tt lpfCnt}}
\def\reset{\mbox{\tt reset}}
\def\pfLoopList{\mbox{\tt pfLoopList}}
\def\tmpresetA{tmpreset}
\def\tmpreset{\mbox{\tt \tmpresetA{}}}
\def\ewiL{\mbox{unique-worker}}
\def\ewiomp{\mbox{UW-OpenMP}}
\def\ewi{\mbox{UW}}
\def\Ss{\mbox{Phase-guard}}
\def\Cb{\mbox{Barrier-phase-guard}}
\def\ss{\mbox{phase-guard}}
\def\cb{\mbox{barrier-phase-guard}}
\def\Mr{\mbox{Macro-replacement}}
\def\mr{\mbox{macro-replacement}}
\def\Nwg{\mbox{Non-while-guard}}
\def\nwg{\mbox{non-while-guard}}
\def\bm{\bmname{}\mbox{-model}}
\def\bmname{\mbox{Base}}
\def\ewigroup{\mbox{UW-group}}
\def\ibs{\mbox{iteration-barrier-set}}
\def\ewiTrans{\mbox{UW-Transformation}}
\def\ont{\mbox{\small \tt OMP\_NUM\_THREADS}}
\def\forh{\mbox{\tt for(i=0;i<N;i++)}}
\def\forha{\mbox{\tt for(i=0;i<4;i++)}}
\def\rlxset{\mbox{$T_{\Pi}^Y$}}
\def\scset{\mbox{$T_{\Pi}^{SC}$}}
\def\th{$^{th}$}
\newcommand{\myreport}[1]{}
\newcommand{\txcolr}[1]{\textcolor{red}{#1}}
\newcommand{\txcolb}[1]{\textcolor{blue}{#1}}
\newcommand{\txcolbb}[1]{\textcolor{blue}{\textbf{#1}}}
\newcommand{\txvanish}[1]{\textcolor{white}{#1}}
\newcommand{\txcolrb}[1]{\textcolor{red}{\textbf{#1}}}
\newcommand{\Eval}[1]{\ensuremath{\llbracket}#1\ensuremath{\rrbracket}}
\newcommand{\Hset}[1]{\ensuremath{\langle}#1\ensuremath{\rangle}}
\newcommand{\asplossubmissionnumber}{289}
\newcommand{\emhead}[1]{$ $\\ \noindent {\bf {\em #1} }}
% The following \documentclass options may be useful:

% preprint      Remove this option only once the paper is in final form.
% 10pt          To set in 10-point type instead of 9-point.
% 11pt          To set in 11-point type instead of 9-point.
% authoryear    To obtain author/year citation style instead of numeric.

%\usepackage{times}



\setbeamertemplate{bibliography item}{[\theenumiv]}


\title{\Large{Memory Consistency Models}}
\author[Raghesh A]{\Large{Raghesh A }(CS12D015)\newline\newline
                   %\footnotesize{Guide: V Krishna Nandivada}
}
\institute{PACE Lab, Department of CSE, IIT Madras}
\date{\today}

\begin{document}



% slide
\begin{frame}
\titlepage
\end{frame}

\begin{frame}{Why ...?}
\begin{itemize}
\item Multithreaded programs run on either
  \begin{itemize}
  \item a Uniprocessor \pause OR
  \item Multicores.
  \end{itemize}
\pause
\item One simple way to argue about the correctness of multithreaded programs
      is by looking at the output produced.
\pause
\item Multiple outputs may be acceptable.
\pause
\item The indeterministic nature of multithreaded programs makes programmer's
      life difficult to ensure correctness.
\pause
  \begin{itemize}
  \item One common approach is to apply synchronization.
  \item But the degree of synchronization must be minimal.
  \item Basically synchronization controls the order of read and write to
        shared memory locations.
  \item Without sychronization the order of read and writes may be altered either
        by compiler (optmizations) or by hardware for better performance.
  \end{itemize}
\pause
\item The objective.
  \begin{itemize}
  \item Less compromise on performance - With less degree of synchronization 
  shall we get acceptable output ????
  \item Arguing about the correctness of the program.
  \end{itemize}

\end{itemize}
\end{frame}

\begin{frame}{Memory Model}
\begin{itemize}
\item A formal specification of how the memory system will appear to the programmer.
\item Eliminating gap between expected behaviour (by the programmer) and the
      actual behaviour.
\item Essentially, a \textbf{Memory Model} defines the legitimate orderings of read and write
      to memory locations.
\pause
  \begin{itemize}
  \item A strict memory model - Easy to argue about correctness, but restricts
        optimizations (both at compiler and hardware level).
  \item A relaxed memory model - More opportunities for optimizations, but not
        easy to argue correctness.
  \end{itemize}
\end{itemize}
\end{frame}

\begin{frame}{Sequential Consistency (SC)}
\begin{itemize}
\item Uniprocessor - A "read" returns the last "write" in program order.
\item Multiprocessor - A "read" may/may not return a value according to "writes" in program order.
  \begin{itemize}
  \item Affected by use of write-buffers, register allocation, ...
  \end{itemize}
\pause
\item \textbf{Sequential consistency} - An intuitive extension of uniprocessor model.
\begin{itemize}
\item \txcolrb{Program order}: Each core executes statements in program order. 
\item Cores are switched in an arbitrary (indeterministic) manner.
\item \txcolrb{Atomicity}: A memory operation executes atomically w.r.t. other memory operations.
\end{itemize}
\pause
\item \underline{An Example}~\cite{Adve2010}\\
\begin{figure}
\small
\centering
Initially $X=Y=0$\\
\begin{tabular} {l | l }
\hline
\txcolr{Red} Thread & \txcolb{Blue} Thread \\
\hline
\txcolr{X = 1;}  & \txcolb{Y = 1;} \\
\txcolr{r1 = Y;} & \txcolb{r2 = X;}\\
\hline
\end{tabular}
\end{figure}
\pause
\item Some possible interleavings.

\begin{figure}
\small
\centering
\begin{tabular} {l | l | l}
\hline
Execution 1 & Execution 2 & Execution 3\\
\hline
\txcolr{X = 1;}       & \txcolb{Y = 1;}       & \txcolr{X = 1;} \\
\txcolr{r1 = Y;}      & \txcolb{r2 = X;}      & \txcolb{Y = 1;}\\
\txcolb{Y = 1;}       & \txcolr{X = 1;}       & \txcolr{r1 = Y;} \\
\txcolb{r2 = X;}      & \txcolr{r1 = Y;}      & \txcolb{r2 = X;}\\
//\txcolr{r1} == 0;   & //\txcolr{r1} == 1;    & //\txcolr{r1} == 1; \\
//\txcolb{r2} == 1;   & //\txcolb{r2} == 0;    & //\txcolb{r2} == 1;\\
\hline
\end{tabular}
\end{figure}

\end{itemize}
\end{frame}

\begin{frame}{Relaxed Consistency (RC)}
\begin{itemize}
\item SC is hard to realize.
\item Too much restriction for hardware and compiler optmizations
\item Even SC may not always produce expected output.
\item So we can live with \textbf{Relaxed Consistency Models}.
\pause
\item Potential relaxations~\cite{rajeev-utah}
  \begin{itemize}
  \item Program order: (Refers to different memory locations)
    \begin{itemize}
    \item Relax $W \rightarrow R$ program order.
    \item Relax $W \rightarrow W$ program order.
    \item Relax $R \rightarrow R$ and $R \rightarrow W$ program order.
    \end{itemize}
\pause
  \item Write atomicity: (Refers to same memory location)
    \begin{itemize}
    \item Read others' write early.
    \end{itemize}
\pause
  \item Write atomicity and program order.
    \begin{itemize}
    \item Read own write early
    \end{itemize}
  \end{itemize}
\end{itemize}
\end{frame}

\begin{frame}{Total Store Order (TSO)}
\underline{\textbf{Definition}}~\cite{rajeev-utah}:\\
\begin{enumerate}

\item A read can complete before an earlier write to a different address.
%\item A read can not return the value of a write by another processor unless all
%      processors have seen the write.
\item A read can return the value of own write before others see it (Makes use of store buffers).
\end{enumerate}
\pause
\begin{figure}
\small
\begin{tabular}{|l|p{1.2cm}|p{1.2cm}|p{1.3cm}|p{1.2cm}|p{1.2cm}|}
\hline
Model & $W \rightarrow R$ order & $W \rightarrow W$ order & $R \rightarrow RW$ order & R Others' W Early & R Own W Early\\
\hline
TSO   & $\surd$                 &                         &                          &                   & $\surd$      \\
\hline
\end{tabular}
\end{figure}
\pause
\begin{itemize}
\item \underline{An Example}\\
\begin{figure}
\small
\centering
Initially $X=Y=0$\\
\begin{tabular} {l | l }
\hline
\txcolr{Red} Thread & \txcolb{Blue} Thread \\
\hline
\txcolr{X = 1;}  & \txcolb{Y = 1;} \\
\txcolr{r1 = Y;} & \txcolb{r2 = X;}\\
\hline
\end{tabular}
\end{figure}
\pause
\item Some possible interleavings (in addition to that of SC).
\begin{minipage}{0.4\textwidth}
\begin{figure}
\centering
\small
\begin{tabular} {|l|l|}
\hline
Time & Execution 1 \\
\hline
$t_1$									& \txcolr{r1 = Y;}   \\
\hline
$t_2 = t_1 + \delta$ 	& \txcolr{X = 1;}    \\
\hline
$t_2 = t_1 + \delta$ 	& \txcolb{Y = 1;}     \\
\hline
$t_3 = t_2 + \delta$  & \txcolb{r2 = X;}   \\
\hline
                     	&//\txcolr{r1} == 0;\\
\hline
                     	&//\txcolb{r2} == 0;\\
\hline
\end{tabular}
\end{figure}
\end{minipage}
\begin{minipage}{0.5\textwidth}
\begin{itemize}
\item Effect of 1: It is possible to reorder read to \txcolr{Y} before write to \txcolr{X}.
\item Effect of 2: It is possible to get value of \txcolb{r2} as 0, because an
      earlier write to \txcolr{X} by \txcolr{Red} thread might have stored in
      store-buffer and is unseen by the \txcolb{Blue} thread.
\end{itemize}
\end{minipage}
\pause
\small
\item r1 = 0 and r2 = 0 is possible in TSO, but not in SC !!!
\end{itemize}

\end{frame}

\begin{frame}{Verification of Memory models}
\underline{Paper 1:}\\
\vspace{0.3cm}
\underline{Title}: \textbf{On the verification problem for Weak Memory Models}\\
\underline{Authors}: {\em Mohamed Faouzi Atig, Ahmed Bouajjani, Sebastian Burckhardt, Mandanlal Musuvanthi.}\\
\underline{Appeared in}: POPL'10\\
\pause
\vspace{0.3cm}
\begin{itemize}
\item Addresses the verification problem of finite-state-concurrent programs running under
      weak memory models.
\pause
\item Important for
	\begin{itemize}
  \item Correctness of concurrency libraries.
  \item Correctness of compiler transformations.
	\end{itemize}
\pause
\item Weak memory models considered.
  \begin{itemize}
  \item Relaxing "write to read" order - TSO (Total Store Order).
  \item Relaxing "write to read" and "write to write" order - PSO (Partial Store Order).
  \item Relaxing "read to read/write" - RMO (Relaxed Memory Ordering).
  \end{itemize}
\pause
\item Abstract operational models for weak memory models are defined based on state
      machines with (potentially unbound) FIFO buffers.
\pause
\item Studied the decidability of thier reachability and repeated reachability
      problems.
\pause
\item Proved that
	\begin{itemize}
  \item \textbf{reachability} problem is \textbf{decidable} for TSO and PSO.
  \item \textbf{reachability} problem is \textbf{undicidable} for RMO.
  \item \textbf{repeated reachability} problem is \textbf{undecidable} for all models.
	\end{itemize}
\end{itemize}
\end{frame}

\begin{frame}{Paper 1: On the verification problem for Weak Memory Models}
\begin{itemize}
\item A formal model of concurrent finite-state programs on a TSO system.
	\begin{itemize}
	\item A finite-state control representing local program states of each thread.
  \item Unbounded FIFO queues storing contents of store buffers.
	\end{itemize}
\pause
\item The decidability problems are studied by establishing a close connection
      between TSO systems and lossy FIFO Channels.
\end{itemize}
\end{frame}

\begin{frame}{Paper 1: On the verification problem for Weak Memory Models}
\underline{\textbf{Definitions and Notations}}:\\

\begin{itemize}
\item Let $k \in \mathbb{N}, k \ge 1$, Then $[k]$ denotes the set of intergers
$\{1,...,k\}$.
\item $\Sigma$ - finite alphabet, $\varepsilon$ - empty word, $\Sigma_\varepsilon$ is $\Sigma \cup \{\varepsilon\}$.
\item $length(w)$ - length of a word, $length(\varepsilon) = 0$.
\item $\forall i \in [lenght(w)]$, $w(i)$ denotes the symbol at position $i$ in $w$.
\item For $a \in \Sigma$ and $w \in \Sigma^{*}$, $a \in w$ if $\exists i \in [length(w)]$ s.t $a=w(i)$.
\pause
\item Let $\Theta \subseteq \Sigma$ and a word $u \in \Sigma^{*}$, the \textbf{projection} of $u$ over $\Theta$, $u|_\Theta$
      is the word obtained from u by erasing all the symbols that are not in $\Theta$.
\pause
\item A \textbf{substitution} $\sigma$ over $\Sigma$ is a mapping from $\Sigma$ to $\Sigma$.
	\begin{itemize}
	\item $w[\sigma]$ is the word s.t. $\forall i \in [length(w)]$, $w[\sigma](i) = \sigma(w(i))$.
  \item Generalizing to sets of words: $\forall L \subseteq \Sigma^*$, $L[\sigma] = \{w[\sigma] | w \in L\}$.
  \end{itemize}
\pause
\item $E$ is a set, $k \ge 1$. Let $\mathbf{e} = (e_1,...,e_k) \in E^k$ be $k$-dim vector over $E$.
      $\forall j \in [k]$ and $e^{\prime} \in E$, $\mathbf{e}[j \hookleftarrow e^{\prime}]$ denotes the $k$-dim vector
      $\mathbf{e}^{\prime}$ over $E$ defined as: $\mathbf{e^{\prime}}[j] = e^{\prime}$ and $\mathbf{e^{\prime}}[l] = \mathbf{e}[l]$ 
      for all $l \neq j$.
\pause 
\item $E$ and $F$ are two sets. $[E \rightarrow F]$ denotes the set of all mappings from $E$ to $F$. 
       Let $E = \{e_1,...,e_k\}$ for some integer $k \ge 1$. Then sometimes a mapping is identified
       $\mathbf{g} \in [E \rightarrow F]$ with a $k$-dim vector over $F$.
       i.e., $\mathbf{g} \in F^k$ with $\mathbf{g}[i] = e_i$ for all $i \in [k]$.
\end{itemize}
\end{frame}

\begin{frame}{Paper 1: On the verification problem for Weak Memory Models}
\underline{\textbf{TSO/PSO concurrent systems}}
\begin{itemize}
\item $D$ - A finite data domain.
\item $X = \{x_1,...,x_m\}$ - A finite set of variables s.t $\forall i, x_i \in D$.
\item $M = D^m$ - The set of all possible valuations of the variables in X.
\item $I$ - Finite set of process identifiers.
\item $\Omega(I,D,X)$ - The smallest set of operations containing
	\begin{enumerate}
	\item $nop$ - "\em{no operation}".
	\item $r(i,x,d)$ - \em{read operations}.
	\item $w(i,x,d)$ - \em{write operations}.
	\item $arw(i,x,d,d^{\prime})$ - \em{atomic read-write operations}.
	\end{enumerate}
	where $i \in I$, $x \in X$, $d,d^{\prime} \in D$.
\pause
\item A \textbf{concurrent} system over $D$ and $X$ is a tuple $\mathpzc{N} = 
      (\mathpzc{P_1},...,\mathpzc{P_n})$, where 
	\begin{itemize}
	\item $i=[n]$,
	\item $\mathpzc{P_i} = (P_i, \Delta_i)$,
	\item $P_i$ - Finite set of control states,
	\item $\Delta_i \subseteq P_i \times \Omega(\{i\}, D, X) \times P_i$ - Finite
	      set of transition rules.
	\item $p \overset{op}{\longrightarrow} p^{\prime}$ implies $(p, op, p^{prime}) \in \Delta_i$,
		    for any $p, p^{prime} \in P_i$ and $op \in \Omega({i}, D, X)$.
	\end{itemize}
\end{itemize}
\end{frame}

\begin{frame}{Paper 1: On the verification problem for Weak Memory Models}
\underline{\textbf{An operational model for TSO}}
\begin{itemize}
\item In TSO the sequence $w(i, y, d^{\prime})r(i,x,d)$ can be reordered to $r(i,x,d)w(i, y, d^{\prime})$
		  provided
			\begin{itemize}
			\item they are performed by same process,
			\item $x \neq y$.
			\end{itemize}
\item For each process a FIFO buffer is associated to store the write operations.
\pause
\item $\mathbf{P} = P_1 \times...\times P_n$.
\item $B_i = \{i\} \times [m] \times D$ - Alphabet of the store for process $\mathpzc{P_i}$, $i \in [n]$.
\pause
\item Configuration $\mathpzc{N}$ - $\langle \mathbf{p}, \mathbf{d}, \mathbf{u} \rangle$, where $\mathbf{p} \in \mathbf{P}$,
	    $\mathbf{d}, \mathbf{d^{\prime}} \in M$ and $\mathbf{u} \in B_1 \times...\times B_n$ (valuation of 
			store buffers).
\end{itemize}
\end{frame}

\begin{frame}{Paper 1: On the verification problem for Weak Memory Models}
\underline{\textbf{An operational model for TSO}}\\
\begin{itemize}
\item \underline{Transition relation $\Longrightarrow_\mathpzc{N}$}
\begin{itemize}
\item $\langle \mathbf{p}, \mathbf{d}, \mathbf{u}\rangle \Longrightarrow_\mathpzc{N} \langle \mathbf{p^{\prime}}, \mathbf{d^{\prime}}, \mathbf{u^{\prime}} \rangle$,
      if there is an $i \in [n]$ and $p, p^{\prime} \in P_i$ s.t $\mathbf{p}[i] = p$,
			$\mathbf{p^{\prime}} = \mathbf{p}[i \hookleftarrow p{\prime}]$ and one of the following holds
			\begin{enumerate}
			\item Nop:\\
			\vspace{2mm}
				    $p \overset{nop}{\longrightarrow}_i p{\prime}$, $\mathbf{d} = \mathbf{d^{\prime}}$, $\mathbf{u} = \mathbf{u^{\prime}}$.
			\vspace{2mm}
			\item Write to store: Store write operation to the FIFO buffer.\\
			\vspace{2mm}
						$p \overset{w(i,x,d)}{\longrightarrow}_i p{\prime}$, $\mathbf{u^{\prime}} = \mathbf{u}[i \hookleftarrow (i,j,d)\mathbf{u}[i]]$
						and $\mathbf{d} = \mathbf{d^{\prime}}$.
			\vspace{2mm}
			\item Update: Choose a process non-deterministically and update memory by executing the first write operation
										in its buffer.\\
			\vspace{2mm}
						$p = p^{\prime}$ and $\exists j \in [m].\exists d \in D.\mathbf{u} = \mathbf{u}^{\prime}[i \hookleftarrow
						\mathbf{u}^{\prime}[i](i,j,d)]$ and $\mathbf{d}^{\prime} = \mathbf{d}[j \hookleftarrow d]$.
			\vspace{2mm}
			\item Read: A read by process $\mathpzc{P}_i$ to variable $x_j$ can overtake write operations in its own store
						buffer if these operations are on other variables than $x_j$.\\
			\vspace{2mm}
						(Read memory): $\forall (i,k,d^{\prime}) \in \mathbf{u}[i].k \neq j$, and $\mathbf{d}[j] = d$.\\
						(Read own write): $\exists u_1, u_2 \in B_i^*.(\mathbf{u}[i] = u_1(i,j,d)u_2$ and
															$\forall (i,k,d^{\prime}) \in u_1.k \neq j)$.
			\vspace{2mm}
			\item ARW:\\
			\vspace{2mm}
			$p \overset{arw(i,x_j,d,d^{\prime}}{\longrightarrow}_i p^{\prime}, \mathbf{u}[i] = \varepsilon, \mathbf{u} = \mathbf{u}^{\prime}$,
					and $\mathbf{d}[j] = d$, and $\mathbf{d}^{\prime} = \mathbf{d}[j \hookleftarrow d^{\prime}])$.
			\end{enumerate}
\end{itemize}
\end{itemize}
\end{frame}

\begin{frame}{Correctness of RC}
\underline{Paper}\\
Effective Program Verification for Relaxed Memory Models, by Sebastian Burckhardt and
Madanlal Musuvanthi~\cite{Burckhardt2008}.
\begin{itemize}
\item Same program may exhibit more executions on
      a relaxed model than SC.
\item Let $T_{\Pi}^Y$ be the set of executions of program $\Pi$ on memory model Y.
        Then $T_{\Pi}^{SC} \subset T_{\Pi}^Y$~.
\item To verify relaxed executions \rlxset{}, verify following two problems.
  \begin{itemize}
  \item Use standard verification methodology for concurrent programs to show that
        the executions in \scset{} are correct.
  \item Use specialized methodology for {\em memory model safety} verification
        showing that \rlxset{} $=$ \scset{}.
  \item The program is $Y-safe$  if \rlxset{} $=$ \scset{}.
  \end{itemize}
\end{itemize}
\end{frame}

\begin{frame}{Correctness of RC~}
\begin{itemize}
\item Generally usage {\em Store buffers} corresponds to TSO.
\item This paper uses TSO-safety and store-buffer-safety interchangably.
\item \textbf{Borderline execution} - SC execution extended into $T_{\Pi}^{TSO} \setminus  T_{\Pi}^{SC}$.
\item A program is store-buffer-safe exactly if there are no borderline executions.
\end{itemize}

\end{frame}

\begin{frame}{Correctness of RC~}
\begin{itemize}
\item Problem formulation.
	\begin{itemize}
	\item {\em A trace} - A collection of events, each representing a memory access
	      (Store, load or interlocked operation).
	\item {\em Issue index} - Sequence number relative to all events by the same processor.
	\item {\em Coherence index} - Sequence number of the value that is read or written
        by the event, relative to the entire value sequence written to the targeted memory 
			  location during the execution.	
	\end{itemize}
\item Formally: \\
\hspace{2cm}$Op = \{st, ld, il\}$, $\mathbb{N} = $set of natural numbers\\
\hspace{2cm}$Proc = {1,...,N}$, finite set of processor ids, $N \in \mathbb{N}$.\\
\hspace{2cm}$Adr$ - fininte set of memory addresses.\\
\hspace{2cm}$\mathbb{N}_0 \subset \mathbb{Z}$ - set of non negative integers

Then\\
\hspace{2cm}$Evt = Op \times Proc \times \mathbb{N} \times Adr \times \mathbb{N}_0$


              
\end{itemize}
\end{frame}

\begin{frame}{Linearisability}
\end{frame}

\footnotesize{
\bibliographystyle{unsrt}
\bibliography{memmodel}
}

\end{document} 
