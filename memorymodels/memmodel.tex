\documentclass[9pt]{beamer}
\usepackage{caption}
\usepackage{subcaption}
\usepackage{semantic}
%\usepackage{stmaryrd}
\usepackage{amsthm}
\usepackage{graphicx}
\usepackage{url}

\usepackage{tikz}
\usepackage{amsmath}
\usepackage{courier}
\usepackage{enumerate}
\usepackage{ragged2e}
\usepackage{fixltx2e}
\setbeamertemplate{itemize items}[circle]
\setbeamertemplate{navigation symbols}{}
%\setbeamertemplate{footline}[frame number]
\setbeamertemplate{footline}{
            \raisebox{5pt}{\makebox[\paperwidth]{\hfill\makebox[10pt]{\scriptsize\insertframenumber}}}}
\usepackage{listings}

\lstdefinestyle{footnote}{mathescape,captionpos=b, basicstyle=\footnotesize\ttfamily,
numbers=left, numberstyle=\tiny, xleftmargin=5em}
\lstdefinestyle{normal}{mathescape,captionpos=b, basicstyle=\normalsize\ttfamily,
numbers=left, numberstyle=\tiny, xleftmargin=5em}
\lstdefinestyle{tiny}{mathescape,captionpos=b, basicstyle=\tiny\ttfamily,
numbers=left, numberstyle=\tiny, xleftmargin=5em}
\lstdefinestyle{footnote1}{mathescape, basicstyle=\footnotesize\ttfamily}
\lstdefinestyle{tiny1}{mathescape, basicstyle=\tiny\ttfamily}
\lstdefinestyle{script1}{mathescape, basicstyle=\scriptsize\ttfamily}


%\captionsetup[subfigure]{labelformat=empty} 
%\DeclareCaptionLabelFormat{no-parens}{\textbf{(#2)}}
%\captionsetup[sub]{              % The subcaption settings
%    font=footnotesize,           % Make the font smaller for both label and text
%    textfont=sl,                 % Make only the caption text slanted
%    labelformat=no-parens}        % Use our custom label format

%\renewcommand\thesubfigure{(\alph{subfigure})}

\def\openmp{\mbox{OpenMP}}
\def\pfloop{\mbox{parallel-for-loop}}
\def\pi{\mbox{agent}}
\def\pis{\pi\mbox{s}}
\def\ompsection{\mbox{parallel-section}}
\def\ompsections{\ompsection\mbox{s}}
\def\pfloops{\pfloop\mbox{s}}
\def\lopenmp{\mbox{LOpenMP}}
\def\canIEnterA{canEnter}
\def\canIEnter{\mbox{\tt \canIEnterA{}}}
\def\iexitA{iExit}
\def\iexit{\mbox{\tt \iexitA{}}}
\def\gcount{\mbox{\tt Rdy}}
\def\grpCnt{\mbox{\tt grCnt}}
\def\pfCnt{\mbox{\tt pfCnt}}
\def\isDoneA{{Done}}
\def\isDone{\mbox{\tt \isDoneA{}}}
\def\lpfCnt{\mbox{\tt lpfCnt}}
\def\reset{\mbox{\tt reset}}
\def\pfLoopList{\mbox{\tt pfLoopList}}
\def\tmpresetA{tmpreset}
\def\tmpreset{\mbox{\tt \tmpresetA{}}}
\def\ewiL{\mbox{unique-worker}}
\def\ewiomp{\mbox{UW-OpenMP}}
\def\ewi{\mbox{UW}}
\def\Ss{\mbox{Phase-guard}}
\def\Cb{\mbox{Barrier-phase-guard}}
\def\ss{\mbox{phase-guard}}
\def\cb{\mbox{barrier-phase-guard}}
\def\Mr{\mbox{Macro-replacement}}
\def\mr{\mbox{macro-replacement}}
\def\Nwg{\mbox{Non-while-guard}}
\def\nwg{\mbox{non-while-guard}}
\def\bm{\bmname{}\mbox{-model}}
\def\bmname{\mbox{Base}}
\def\ewigroup{\mbox{UW-group}}
\def\ibs{\mbox{iteration-barrier-set}}
\def\ewiTrans{\mbox{UW-Transformation}}
\def\ont{\mbox{\small \tt OMP\_NUM\_THREADS}}
\def\forh{\mbox{\tt for(i=0;i<N;i++)}}
\def\forha{\mbox{\tt for(i=0;i<4;i++)}}
\def\rlxset{\mbox{$T_{\Pi}^Y$}}
\def\scset{\mbox{$T_{\Pi}^{SC}$}}
\def\th{$^{th}$}
\newcommand{\myreport}[1]{}
\newcommand{\txcolr}[1]{\textcolor{red}{#1}}
\newcommand{\txcolb}[1]{\textcolor{blue}{#1}}
\newcommand{\txcolbb}[1]{\textcolor{blue}{\textbf{#1}}}
\newcommand{\txvanish}[1]{\textcolor{white}{#1}}
\newcommand{\txcolrb}[1]{\textcolor{red}{\textbf{#1}}}
\newcommand{\Eval}[1]{\ensuremath{\llbracket}#1\ensuremath{\rrbracket}}
\newcommand{\Hset}[1]{\ensuremath{\langle}#1\ensuremath{\rangle}}
\newcommand{\asplossubmissionnumber}{289}
\newcommand{\emhead}[1]{$ $\\ \noindent {\bf {\em #1} }}
% The following \documentclass options may be useful:

% preprint      Remove this option only once the paper is in final form.
% 10pt          To set in 10-point type instead of 9-point.
% 11pt          To set in 11-point type instead of 9-point.
% authoryear    To obtain author/year citation style instead of numeric.

%\usepackage{times}



\setbeamertemplate{bibliography item}{[\theenumiv]}


\title{\Large{Memory Consistency Models}}
\author[Raghesh A]{\Large{Raghesh A }(CS12D015)\newline\newline
                   %\footnotesize{Guide: V Krishna Nandivada}
}
\institute{PACE Lab, Department of CSE, IIT Madras}
\date{\today}

\begin{document}

% slide
\begin{frame}
\titlepage
\end{frame}

\begin{frame}{Why ...?}
\begin{itemize}
\item Multithreaded programs run on either
  \begin{itemize}
  \item a Uniprocessor \pause OR
  \item Multicores.
  \end{itemize}
\pause
\item One simple way to argue about the correctness of multithreaded programs
      is by looking at the output produced.
\pause
\item Multiple outputs may be acceptable.
\pause
\item The indeterministic nature of multithreaded programs makes programmer's
      life difficult to ensure correctness.
\pause
  \begin{itemize}
  \item One common approach is to apply synchronization.
  \item But the degree of synchronization must be minimal.
  \item Basically synchronization controls the order of read and write to
        shared memory locations.
  \item Without sychronization the order of read and writes may be altered either
        by compiler (optmizations) or by hardware for better performance.
  \end{itemize}
\pause
\item The objective.
  \begin{itemize}
  \item Less compromise on performance - With less degree of synchronization 
  shall we get acceptable output ????
  \item Arguing about the correctness of the program.
  \end{itemize}

\end{itemize}
\end{frame}

\begin{frame}{Memory Model}
\begin{itemize}
\item A formal specification of how the memory system will appear to the programmer.
\item Eliminating gap between expected behaviour (by the programmer) and the
      actual behaviour.
\item Essentially, a \textbf{Memory Model} defines the legitimate orderings of read and write
      to memory locations.
\pause
  \begin{itemize}
  \item A strict memory model - Easy to argue about correctness, but restricts
        optimizations (both at compiler and hardware level).
  \item A relaxed memory model - More opportunities for optimizations, but not
        easy to argue correctness.
  \end{itemize}
\end{itemize}
\end{frame}

\begin{frame}{Sequential Consistency (SC)}
\begin{itemize}
\item Uniprocessor - A "read" returns the last "write" in program order.
\item Multiprocessor - A "read" may/may not return a value according to "writes" in program order.
  \begin{itemize}
  \item Affected by cache coherency, use of write-buffers, register allocation, ...
  \end{itemize}
\pause
\item \textbf{Sequential consistency} - An intuitive extension of uniprocessor model.
\begin{itemize}
\item \txcolrb{Program order}: Each core executes statements in program order. 
\item Cores are switched in an arbitrary (indeterministic) manner.
\item \txcolrb{Atomicity}: A memory operation executes atomically w.r.t. other memory operations.
\end{itemize}
\pause
\item \underline{An Example}~\cite{Adve2010}\\
\begin{figure}
\small
\centering
Initially $X=Y=0$\\
\begin{tabular} {l | l }
\hline
\txcolr{Red} Thread & \txcolb{Blue} Thread \\
\hline
\txcolr{X = 1;}  & \txcolb{Y = 1;} \\
\txcolr{r1 = Y;} & \txcolb{r2 = X;}\\
\hline
\end{tabular}
\end{figure}
\pause
\item Some possible interleavings.

\begin{figure}
\small
\centering
\begin{tabular} {l | l | l}
\hline
Execution 1 & Execution 2 & Execution 3\\
\hline
\txcolr{X = 1;}       & \txcolb{Y = 1;}       & \txcolr{X = 1;} \\
\txcolr{r1 = Y;}      & \txcolb{r2 = X;}      & \txcolb{Y = 1;}\\
\txcolb{Y = 1;}       & \txcolr{X = 1;}       & \txcolr{r1 = Y;} \\
\txcolb{r2 = X;}      & \txcolr{r1 = Y;}      & \txcolb{r2 = X;}\\
//\txcolr{r1} == 0;   & //\txcolr{r1} == 1;    & //\txcolr{r1} == 1; \\
//\txcolb{r2} == 1;   & //\txcolb{r2} == 0;    & //\txcolb{r2} == 1;\\
\hline
\end{tabular}
\end{figure}

\end{itemize}
\end{frame}

\begin{frame}{Relaxed Consistency (RC)}
\begin{itemize}
\item SC is hard to realize.
\item Too much restriction for hardware and compiler optmizations
\item Even SC may not always be produce correct output.
\item So we can live with \textbf{Relaxed Consistency Models}.
\pause
\item Potential relaxations~\cite{rajeev-utah}
  \begin{itemize}
  \item Program order: (Refers to different memory locations)
    \begin{itemize}
    \item Relax $W \rightarrow R$ program order.
    \item Relax $W \rightarrow W$ program order.
    \item Relax $R \rightarrow R$ and $R \rightarrow W$ program order.
    \end{itemize}
\pause
  \item Write atomicity: (Refers to same memory location)
    \begin{itemize}
    \item Read others' write early.
    \end{itemize}
\pause
  \item Write atomicity and program order.
    \begin{itemize}
    \item Read own write early
    \end{itemize}
  \end{itemize}
\end{itemize}
\end{frame}

\begin{frame}{Total Store Order (TSO)}
\underline{\textbf{Definition}}~\cite{rajeev-utah}:\\
\begin{enumerate}
\item A read can complete before an earlier write to a different address.
\item A read can not return the value of a write by another processor unless all
      processors have seen the write.
\item A read can return the value of own write before others see it (Makes use of store buffers).
\end{enumerate}
\pause
\begin{figure}
\small
\begin{tabular}{|l|p{1.2cm}|p{1.2cm}|p{1.3cm}|p{1.2cm}|p{1.2cm}|}
\hline
Model & $W \rightarrow R$ order & $W \rightarrow W$ order & $R \rightarrow RW$ order & R Others' W Early & R Own W Early\\
\hline
TSO   & $\surd$                 &                         &                          &                   & $\surd$      \\
\hline
\end{tabular}
\end{figure}
\pause
\begin{itemize}
\item \underline{An Example}\\
\begin{figure}
\small
\centering
Initially $X=Y=0$\\
\begin{tabular} {l | l }
\hline
\txcolr{Red} Thread & \txcolb{Blue} Thread \\
\hline
\txcolr{X = 1;}  & \txcolb{Y = 1;} \\
\txcolr{r1 = Y;} & \txcolb{r2 = X;}\\
\hline
\end{tabular}
\end{figure}
\pause
\item Some possible interleavings (in addition to that of SC).
\begin{minipage}{0.4\textwidth}
\begin{figure}
\centering
\small
\begin{tabular} {|l|l|}
\hline
Time & Execution 1 \\
\hline
$t_1$ & \txcolr{r1 = Y;}   \\
\hline
$t_2 = t_1 + \delta$ & \txcolr{X = 1;}    \\
\hline
$t_2 = t_1 + \delta$ & \txcolb{Y = 1;}     \\
\hline
$t_3 = t_2 + \delta$  & \txcolb{r2 = X;}   \\
\hline
                     &//\txcolr{r1} == 0;\\
\hline
                     &//\txcolb{r2} == 1;\\
\hline
\end{tabular}
\end{figure}
\end{minipage}
\begin{minipage}{0.5\textwidth}
\begin{itemize}
\item Effect of 1: It is possible to reorder read to \txcolr{Y} before write to \txcolr{X}.
\item Effect of 2: It is impossible to get value of \txcolb{r2} as 0, by reading 0 from \txcolb{X},
      because an earlier write to \txcolr{X} by \txcolr{Red} thread should have seen by the \txcolb{Blue}
      thread.
\end{itemize}
\end{minipage}
\pause
\small
\item r1 = 0 and r2 = 0 is possible in TSO, but not in SC !!!
\end{itemize}

\end{frame}

\begin{frame}{Correctness of RC~\cite{Burckhardt2008}}
\begin{itemize}
\item Same program may exhibit more executions on
      a relaxed model than SC~\cite{Burckhardt2008}.
\item Let $T_{\Pi}^Y$ be the set of executions of program $\Pi$ on memory model Y.
        Then $T_{\Pi}^Y \subset T_{\Pi}^{SC}$~\cite{Burckhardt2008}.
\item To verify relaxed executions \rlxset{}, verify following two problems.
  \begin{itemize}
  \item Use standard verification methodology for concurrent programs to show that
        the executions in \scset{} are correct.
  \item Use specialized methodology for {\em memory model safety} verification
        showing that \rlxset{} $=$ \scset{}.
  \item The program is $Y-safe$  if \rlxset{} $=$ \scset{}.
  \end{itemize}
\end{itemize}
\end{frame}

\begin{frame}{Linearisability}
\end{frame}




\footnotesize{
\bibliographystyle{unsrt}
\bibliography{memmodel}
}

\end{document} 
